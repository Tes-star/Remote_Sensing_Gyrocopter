% ----------------------- Vorlage einlesen ---------------------------
% Dokumentenklasse festlegen
\documentclass[a4paper, 12pt, chapterprefix=false]{scrreprt}

% ----------------------- Packages importieren ---------------------------

\usepackage[utf8]{inputenc}
\usepackage[ngerman]{babel}
\usepackage{amsmath}
\usepackage{amssymb}
\usepackage{fancyhdr}
\usepackage{color}
\usepackage{graphicx}
\usepackage{lastpage}
\usepackage{pdflscape}
\usepackage{subfigure}
\usepackage{float}
\usepackage{polynom}
\usepackage{forloop}
\usepackage[a4paper]{geometry}
\usepackage{listings}
\usepackage{fancybox}
\usepackage{tikz}
\usepackage{algpseudocode,algorithm,algorithmicx}
\usepackage[onehalfspacing]{setspace}
\usepackage{booktabs}
\usepackage{multirow}
\usepackage{tabularx}
\usepackage{hyperref}
\usepackage{blindtext}
\usepackage{pdfpages}
\usepackage{enumitem}
\usepackage{parskip}




% ----------------------- Seitenlayout ---------------------------

% Größe der Ränder setzen
\geometry{a4paper, left=2.5cm, right=2.5cm, top=2.5cm, bottom=2.5cm}
\parindent= 0pt

% Kopfzeile 
\pagestyle {fancy}
\fancyhead[L]{\Modul}
\fancyhead[C]{Team Gyrocopter}
\fancyhead[R]{\today}

% Fußzeile
\fancyfoot[L]{}
\fancyfoot[C]{}
\fancyfoot[R]{Seite \thepage /\pageref*{LastPage}}


% ----------------------- Layout Überschriften / Abschnitte ---------------------------

\renewcommand*\chapterheadstartvskip{\vspace*{-\topskip}}
\renewcommand*\chapterheadendvskip{
\vspace*{1\baselineskip plus .1\baselineskip minus .167\baselineskip}}



\setkomafont{chapter}{\fontsize{17bp}{18.8bp}\selectfont\bfseries}
\setkomafont{section}{\fontsize{15bp}{18.8bp}\selectfont\bfseries}
\setkomafont{subsection}{\fontsize{13bp}{18.8bp}\selectfont\bfseries}


% Abstand Auflistung ändern
\setitemize{itemsep= -2pt}
\setenumerate{itemsep= -2pt}
\setstretch{1.15}
\setlength{\parskip}{1pt}
% ----------------------- Erstellung neuer Kommandos ---------------------------
\newcommand{\Heading}{Team Gyrcopter}
\newcommand{\Modul}{Projekt Data Science}

\newcommand{\Dozent}{Dozent: Prof. Dr. Christian Hänig}


% ----------------------- Beginn Dokument ---------------------------
\begin{document}
% ----------------------- Vorlage Titelseite einlesen ---------------------------
\begin{titlepage}


\begin{figure}[h]
\begin{center}
\includegraphics[width=10cm]{../../data/HSA_Logo}
\end{center}\label{fig:hsa_logo_figure}
\end{figure}

\begin{center}

\vspace{3.5 cm}


{\Huge \textbf{\Heading}}

\vspace{1.3 cm}

{\Huge \textbf{Modul: \Modul}}\\

\vspace{3 cm}

\begin{onehalfspace}


\begin{large}

\Semester \\


\Dozent \\


\vfill




Gruppe: Timo van den Wyenbergh und Felix Graß\\

Studiengang Data Science M.Sc.\\

Sommersemester 2022\\


\end{large}
\end{onehalfspace}

\end{center}


\end{titlepage}

\newpage % ----------------------- Neue Seite ---------------------------


\tableofcontents % Inhaltsverzeichnis

\thispagestyle{empty}

\newpage % ----------------------- Neue Seite ---------------------------
\setcounter{page}{1} % Blattnummerierung

\chapter{Konzept}\thispagestyle{fancy}

\section{Kontext}

Bei der Findung eines geeigneten Themas für die Projektarbeit des Modul \"Projekt Data Science\" suchten wir
nach einer Problemstellung im Bereich der Bildverarbeitung von Luftaufnahmen.
In diesem Zusammenhang lernten wir Herrn Prof. Dr. Bannehr vom Fachbereich 3 der Hochschule Anhalt kennen.
Herr Bannehr befasst sich mit dem Aufgabengebiet der Geodatenerfassung und besitzt am Fachbereich unter anderem einen
Gyrocopter, mit dem er bereits mehrere Befliegungen durchführte.
Für unsere Projektarbeit erhielten wir von Herrn Bannehr die Luftaufnahmen von Oldenburg und Dessau und überlegten uns
nach Begutachtung der Datengrundlage eine geeignete Aufgabenstellung.

\section{Datengrundlage}
Die Datengrundlage umfasste die Luftaufnahmen der Befliegung von Oldenburg und Dessau.
Dabei handelte es sich bei den Datenformaten um Hyperspektral-, Thermal- und Höhenmeter-Daten.
Leider wurde bei der Befliegung von Dessau aber keine Thermaldaten aufgenommen und weil das übermittelte Gesamtbild
zusätzliche Prozessierungsfehler erhielten verwendeten wir nur die Daten von Oldenburg.

\section{Aufgabenstellung}
Bei der anfänglichen Aufgabenstellung von Herrn Bannehr handelte es sich um einen Klassifikationsvergleich von Dachmaterialien.
Geplant war der Vergleich der Ergebnisse der Methoden des Fachbereichs 3 zu unseren Ergebnissen mit mehr individualisierbaren Methoden.
Da sich nach den ersten Wochen der Bearbeitung herausstellte, dass es keine annotierten Daten gibt und keine Vergleichsarbeit gibt,
änderten wir unter Absprach mit Herrn Bannehr das Thema.\\
Die neue Aufgabenstellung befasste sich nun mit der Bildsegmentierung der Luftaufnahme in acht Klassen, die fast vollständig
das Bild beschreiben. Zu den klassifizieren Instanzen gehören Wiese, Wald, Schienen, Straße, Auto, See, Häuser und None.





\newpage % ----------------------- Neue Seite ---------------------------

\chapter{Projektzusammenfassung}\thispagestyle{fancy}

\newpage % ----------------------- Neue Seite ---------------------------



\newpage % ----------------------- Neue Seite ---------------------------



\end{document}
